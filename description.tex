\documentclass[a4paper]{article}

\usepackage[colorlinks=true, pdfstartview=FitV, linkcolor=blue,citecolor=blue, urlcolor=blue]{hyperref}

\title{Improving image classification with CNNs by exploiting
  selectivity in search \& training data}
\author{Muhammad Iqbal Tawakal\\
 Supervisor: Josephine Sullivan}
\date{}

\begin{document}

\maketitle

\section{Problem: Image classification}

The problem we will tackle in this project is image
classification. Given an image predict a set of labels (from a
pre-defined list) that correspond to the objects in the image. This
multi-class classification problem is typically solved using a set of
binary classifiers. Each classifier predicts the presence or absence
of a particular object. We will use the Image Classification
Competition dataset from
\href{http://pascallin.ecs.soton.ac.uk/challenges/VOC/}{PASCAL Visual
  Object Classes datasets} to assess the classifiers we develop.

\section{Background: Image classification with CNNs}

Deep learning methods, in particular convolutional neural networks
(CNNs)\cite{LeCun:ieee:98} trained in a supervised fashion, are
presently the most powerful approach to image classification problems
\cite{Krizhevsky:nips:12} (absolute performance gains of up to 25\%
over the previous state-of-the-art methods of 3 years ago). And in
this project we will investigate two strategies to improve the
performance of CNNs for the task of image classification, in line with
research already started at CVAP
\cite{Razavian:arxiv:14,Hossein:arxiv:14}. Before describing the
strategies we will employ we will introduce some background detail.


A by-product of training a large and deep CNN to perform image
classification is that you also learn powerful image
representations. The image representation corresponds to the responses
of the network at a layer prior to the output layer. Such an image
representation, if the CNN is sufficiently large and appropriately
trained, can be used to solve very many visual recognition problems
\cite{Razavian:arxiv:14} with the addition of a simple linear
classifier such as an SVM. The first task we assume 


%% CNNs have highlighted the benefit of coupling the learning of the
%% hierarchical feature descriptors for images with learning the
%% classifier.  But why have neural-networks learnt with the
%% back-propagation algorithm, an idea and training algorithm that have
%% been around for decades, suddenly become so effective? The succinct
%% answer is: data, depth and GPUs.  Firstly, the amount of labeled
%% training data available for training has exploded. For example in the
%% \href{http://www.image-net.org}{ImageNet} database there are over 1
%% million images labelled with instances from 1000 visual classes
%% \cite{Russakovsky:iccv:13}.  This data combined with regularization
%% reduces the risk of overfitting. Next it was realized that deep
%% networks with many layers are a much more efficient way to represent
%% complex functions, in terms of the number of parameters needed, than
%% broad shallow networks \cite{Bengio:pami:13}. Thus a manageable number
%% of parameters can represent a very flexible class of functions. The
%% final piece of the puzzle are GPUs. The parameters of the network are
%% learnt by back-propagation (involving convolutions and matrix
%% multiplications) and an iterative stochastic gradient descent
%% algorithm applied with potentially very slow learning rates. Training
%% requires an enormous number of iterations as one makes multiple passes
%% through the labeled data. It is not possible to perform all the
%% necessary computations in a feasible time with CPUs but it is feasible
%% with GPU implementations, like
%% \href{http://code.google.com/p/cuda-convnet/}{\texttt{cuda-convnet}}.


%% \begin{itemize}
%% \item 
%% \href{http://arxiv.org/abs/1403.6382}{CNN Features off-the-shelf: an Astounding Baseline for Recognition}

%% \item 
%% \href{http://www.cs.berkeley.edu/~rbg/papers/r-cnn-cvpr.pdf}{Rich feature hierarchies for accurate object detection and semantic segmentation}
%% \end{itemize}

\section{Research Agenda}

Our project will focus on two somewhat complementary but related
investigations. In the first case we will assume that we have a fixed
CNN image representation and try to find if and where a particular
object type is in image. The hope is that performing this search will
improve image classification.  The second focus will be on learning a
better CNN image representation by mining a large dataset for new
positive and negative examples of a class. These ideas are now
explained in more detail.


\subsection{Selective Search}

Normally, one extracts a single CNN representation for the whole
image. But what happens if the object of interest occupies only a
small portion of the image? The representation is likely to be
dominated by the other larger objects in the image. A better strategy
is to consider many different sub-patch(es) of the image that may
correspond to the object and then extract a CNN representation from
each patch and apply the classifier to each one. The image is then
classified as containing the object of interest if one of the
sub-patches produces a positive result. The strategy of generating
object sub-regions is termed selective search. Luckily there exist
several algorithms to quickly and reliably highlight such sub-patches
\cite{Sande:iccv:11}. These region selection algorithms typically
generate $~$2000 sub-patches while still guaranteeing with a high
probability that the objects in the image are represented by a subset
of the extracted patches.

Recent results in the literature \cite{Girshick:cvpr:14} show that
performing selective search does improve classification accuracy. Our
goal is to re-confirm this result and quantify by how much it improves
results.

\subsection{Fine-tuning the CNN representation with a large number
    of selected training examples} 

It is an empirical finding that training a CNN with larger and more
diverse data produces a more powerful classifier
\cite{Bissacco:iccv:13,Sermanet:iclr:13}. With current hardware,
however, it is not possible to use the whole ImageNet database
$\sim$20 million images during training. To give an idea of why this
is not possible it takes using state-of-the-art GPUs $~$3 weeks to
train a large network $\sim$50 million parameters with 1 million
images from ImageNet. Therefore we propose something in the spirit of:
    \begin{enumerate}
    \item Train a CNN using 1 million images from ImageNet. (We
      already have access to such pre-trained CNNs.)
    %% \item Train a set of simple classifiers to identify the visual
    %%   objects defined by PASCAL using the CNN image representation. 
    \item Apply the CNN to sub-patches extracted from a random subset
      of the images in ImageNet not used during training. We assume
      these patches are labelled. Those that are not confidently
      correctly classified by the CNN are recorded.
    \item Fine-tune the network with these mis-classified examples
      using iterations of the back-propagation in online mode.
    \item Stop if the performance of the network saturates on a
      validation set. Otherwise return to step 2.
    \end{enumerate}
It has been shown that fine-tuning with appropriate data can help
\cite{Agrawal:eccv:14} for the tasks of object detection and fine
grained classification. However, it has not been shown how much it
can help when it is allied to selectively choosing the training data. 


\subsection{Summary of research agenda}
To summarize in this project we will be investigate if these two procedures
\begin{enumerate}
  \item selective search
  \item fine-tuning a CNN with a large number of selected training examples
\end{enumerate}
can improve the accuracy of image classification using CNNs and if
they do to quantify by how much. 

\section{Time Plan for the research}

\subsection{Selective search}
\begin{itemize}
  \item Install software.
  \item Extract CNN feature descriptions from object proposal regions
    for both the test and training images.
  \item Train the set of binary classifiers using the object proposal training regions. 
  \item Apply SVMs to the test set and aggregate the results to
    perform image classification.
\end{itemize}


\subsection{Selective fine-tuning}

\begin{itemize}
  \item Mine the ImageNet database for hard positive and negative examples.
  \item Update the parameters of the CNN with these new examples using
    stochastic gradient descent and then repeat these two tasks.
\end{itemize}



\bibliographystyle{plain}
\bibliography{project}

\end{document}







